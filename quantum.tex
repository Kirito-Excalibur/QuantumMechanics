\documentclass{article}

% Language setting
% Replace `english' with e.g. `spanish' to change the document language
\usepackage[english]{babel}

% Set page size and margins
% Replace `letterpaper' with `a4paper' for UK/EU standard size
\usepackage[letterpaper,top=2cm,bottom=2cm,left=3cm,right=3cm,marginparwidth=1.75cm]{geometry}

% Useful packages
\usepackage{amsmath}
\usepackage{graphicx}
\usepackage[colorlinks=true, allcolors=blue]{hyperref}

\title{Quantum Mechanics}
\author{Nikhil Simon Toppo (IMH/10021/22)}
\date{}
\begin{document}
\maketitle

\newpage

\section*{Introduction}

An understanding of how individual atoms interact with one another to endow macroscopic aggregates of matter with the
physical and chemical properties we observe.
A more general approach to atomic phenomena
is required. Such an approach was developed in 1925 and 1926 by Erwin Schrödinger,
Werner Heisenberg, Max Born, Paul Dirac, and others under the apt name of quantum
mechanics. 

\section{Quantum Mechanics}
In classical mechanics, the future history of a particle is completely determined by its initial position and momentum together with the
forces that act upon it. Quantum mechanics also arrives at relationships between observable quantities, but
the uncertainty principle suggests that the nature of an observable quantity is different in the atomic realm. Cause and effect are still related in quantum mechanics, but
what they concern needs careful interpretation. In quantum mechanics the kind of certainty about the future characteristic of classical mechanics is impossible because the
initial state of a particle cannot be established with sufficient accuracy. The quantities whose relationships quantum mechanics explores are probabilities.


\subsection{Wave Function} 

The quantity with which quantum mechanics is concerned is the wave function $\psi$ of a body. While $\psi$ itself has no physical interpretation, the
square of its absolute magnitude $|\psi|^2$ 
evaluated at a particular place at a particular time
is proportional to the probability of finding the body there at that time.\vspace{10pt}
Wave functions are usually complex with both real and imaginary parts. A probability, however, must be a positive real quantity. The probability density $|\psi|^2$
for a complex $|\psi|^2$ is therefore taken as the product $\psi*\psi$ of $\psi$ and its complex conjugate $\psi*$. 

\subsection*{Wave Function}

\begin{equation}
    \psi=A+iB
\end{equation}

\subsection*{Complex conjugate}

\begin{equation}
    \psi=A-iB
\end{equation}

and so \begin{equation}
    |\psi|^2=\psi*\psi=A^2-i^2B^2=A^2+B^2
\end{equation}

since $i^2=-1$. Hence $|\psi|^2==\psi*\psi$ is always a positive real quantity.

\subsection*{Normalization}

$|\psi|^2$ is proportional to the probability density P of finding the body described by $\psi$, the integral of $|\psi|^2$ over all space must be finite—the body is somewhere, after all. If

\begin{equation}
    \int_{-\infty}^{\infty}|\psi|^2dv=0
\end{equation}
the particle doesn't exist.
\vspace{10pt}
\\
If

\begin{equation}
    \int_{-\infty}^{\infty}|\psi|^2dv=1
\end{equation}

since if the particle exists somewhere at all times,
\begin{equation}
    \int_{-\infty}^{\infty}Pdv=1
\end{equation}

A wave function that obeys Eq.(5) is said to be normalized. Every acceptable wave function must be normalized by multiplying it by an appropriate constant.

\subsection*{Well-Behaved Wave Functions}
$\psi$ must be single-valued. $\delta\psi/\delta x$,$\delta\psi/\delta y$,$\delta\psi/\delta z$ be infinite, continious and single-valued. To summarize:

\begin{enumerate}
    \item $\psi$ must be continous and single-valued everywhere.
    \item $\delta\psi/\delta x$,$\delta\psi/\delta y$,$\delta\psi/\delta z$ must be continous and single-valued everywhere.
    \item $\psi$ must be normizable, which means that $\psi$ must go to 0 as x--$>$$\pm\infty$, y--$>$ $\pm\infty$, z--$>$ $\pm\infty$ in order that $\int |\psi|^2 dV$ over all space be finite constant.
    

\end{enumerate}



For a particle restricted to motion in the x direction, the probability of finding it between $x_1$ and $x_2$ is given by

\subsection{Probability}

\begin{equation}
    P_{{x1}{x2}}=\int_{x1}^{x2}|\psi|^2dx
\end{equation}

\subsection{Wave Equation}

\begin{equation}
    \frac{\delta^2y}{\delta x^2}=\frac{1}{v^2}\frac{\delta^2y}{\delta t^2}
\end{equation}

All solutions must be of the form

\begin{equation}
    y=F(t\pm\frac{x}{v})
\end{equation}

Solution of equation (8) can be expressed as 

\begin{equation}
    y=Acos\omega(t-\frac{x}{v})-iAsin\omega(t-\frac{x}{v})
\end{equation}


\section{Schrodinger Equation: Time-Independent Form}  

In quantum mechanics the wave function, $\psi$ corresponds to the wave variable y of wave function in general.

\subsection*{Free Particle}
\begin{equation}
    \psi=Ae^{i(kx-\omega t)}
\end{equation}
\\
Equation(11) describes the wave equivalent of an unrestricted particle of total energy E and momentum p moving in the +x direction.



\end{document}