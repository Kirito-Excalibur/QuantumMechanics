\documentclass{article}
\usepackage{physics}
% Language setting
% Replace `english' with e.g. `spanish' to change the document language
\usepackage[english]{babel}

% Set page size and margins
% Replace `letterpaper' with `a4paper' for UK/EU standard size
\usepackage[letterpaper,top=2cm,bottom=2cm,left=3cm,right=3cm,marginparwidth=1.75cm]{geometry}

% Useful packages
\usepackage{amsmath}
\usepackage{graphicx}
\usepackage[colorlinks=true, allcolors=blue]{hyperref}

\title{Quantum Mechanics}
\author{Nikhil Simon Toppo (IMH/10021/22)}
\date{}
\begin{document}
\maketitle

\newpage

\section*{Introduction}

An understanding of how individual atoms interact with one another to endow macroscopic aggregates of matter with the
physical and chemical properties we observe.
A more general approach to atomic phenomena
is required. Such an approach was developed in 1925 and 1926 by Erwin Schrödinger,
Werner Heisenberg, Max Born, Paul Dirac, and others under the apt name of quantum
mechanics. 

\section{Quantum Mechanics}
In classical mechanics, the future history of a particle is completely determined by its initial position and momentum together with the
forces that act upon it. Quantum mechanics also arrives at relationships between observable quantities, but
the uncertainty principle suggests that the nature of an observable quantity is different in the atomic realm. Cause and effect are still related in quantum mechanics, but
what they concern needs careful interpretation. In quantum mechanics the kind of certainty about the future characteristic of classical mechanics is impossible because the
initial state of a particle cannot be established with sufficient accuracy. The quantities whose relationships quantum mechanics explores are probabilities.


\subsection{Wave Function} 

The quantity with which quantum mechanics is concerned is the wave function $\psi$ of a body. While $\psi$ itself has no physical interpretation, the
square of its absolute magnitude $|\psi|^2$ 
evaluated at a particular place at a particular time
is proportional to the probability of finding the body there at that time.\vspace{10pt}
Wave functions are usually complex with both real and imaginary parts. A probability, however, must be a positive real quantity. The probability density $|\psi|^2$
for a complex $|\psi|^2$ is therefore taken as the product $\psi*\psi$ of $\psi$ and its complex conjugate $\psi*$. 

\subsection*{Wave Function}

\begin{equation}
    \psi=A+iB
\end{equation}

\subsection*{Complex conjugate}

\begin{equation}
    \psi=A-iB
\end{equation}

and so \begin{equation}
    |\psi|^2=\psi*\psi=A^2-i^2B^2=A^2+B^2
\end{equation}

since $i^2=-1$. Hence $|\psi|^2==\psi*\psi$ is always a positive real quantity.

\subsection*{Normalization}

$|\psi|^2$ is proportional to the probability density P of finding the body described by $\psi$, the integral of $|\psi|^2$ over all space must be finite—the body is somewhere, after all. If

\begin{equation}
    \int_{-\infty}^{\infty}|\psi|^2dv=0
\end{equation}
the particle doesn't exist.
\vspace{10pt}
\\
If

\begin{equation}
    \int_{-\infty}^{\infty}|\psi|^2dv=1
\end{equation}

since if the particle exists somewhere at all times,
\begin{equation}
    \int_{-\infty}^{\infty}Pdv=1
\end{equation}

A wave function that obeys Eq.(5) is said to be normalized. Every acceptable wave function must be normalized by multiplying it by an appropriate constant.

\subsection*{Well-Behaved Wave Functions}
$\psi$ must be single-valued. $\delta\psi/\delta x$,$\delta\psi/\delta y$,$\delta\psi/\delta z$ be infinite, continious and single-valued. To summarize:

\begin{enumerate}
    \item $\psi$ must be continous and single-valued everywhere.
    \item $\delta\psi/\delta x$,$\delta\psi/\delta y$,$\delta\psi/\delta z$ must be continous and single-valued everywhere.
    \item $\psi$ must be normizable, which means that $\psi$ must go to 0 as x--$>$$\pm\infty$, y--$>$ $\pm\infty$, z--$>$ $\pm\infty$ in order that $\int |\psi|^2 dV$ over all space be finite constant.
    

\end{enumerate}



For a particle restricted to motion in the x direction, the probability of finding it between $x_1$ and $x_2$ is given by

\subsection{Probability}

\begin{equation}
    P_{{x1}{x2}}=\int_{x1}^{x2}|\psi|^2dx
\end{equation}

\subsection{Wave Equation}

\begin{equation}
    \frac{\delta^2y}{\delta x^2}=\frac{1}{v^2}\frac{\delta^2y}{\delta t^2}
\end{equation}

All solutions must be of the form

\begin{equation}
    y=F(t\pm\frac{x}{v})
\end{equation}

Solution of equation (8) can be expressed as 

\begin{equation}
    y=Acos\omega(t-\frac{x}{v})-iAsin\omega(t-\frac{x}{v})
\end{equation}


\section{Schrodinger Equation: Time-Independent Form}  

In quantum mechanics the wave function, $\psi$ corresponds to the wave variable y of wave function in general.

\subsection*{Free Particle}
\begin{equation}
    \psi=Ae^{i(kx-\omega t)}
\end{equation}
\\
Equation(11) describes the wave equivalent of an unrestricted particle of total energy E and momentum p moving in the +x direction. We are most interested in situations where the motion of
a particle is subject to various restrictions. An important concern, for example, is an
electron bound to an atom by the electric field of its nucleus. This equation, which is Schrödinger’s equation, can be arrived
at in various ways, but it cannot be rigorously derived from existing physical principles: The equation represents something new.

\subsection*{Time-dependent Schrodinger equation in one dimension}

\begin{equation}
    i\hbar \frac{\delta\psi}{\delta t}= -\frac{\hbar^2}{2m}\frac{\delta^2\psi}{\delta x^2}+U\psi
\end{equation}
\\
U is some function of x,y,z and t.\vspace{10pt}
\\
Schrodinger equation cannot be derived from other basic principles of physics; it is a basic principle in itself.\vspace{10pt}
\\
Newton's second law of motion $F = ma$, the basic principle of classical mechanics, can be derived from
Schrödinger's equation provided the quantities it relates are understood to be averages
rather than precise values. (Newton's laws of motion were also not derived from any
other principles. Like Schrödinger's equation, these laws are considered valid in their
range of applicability because of their agreement with experiment.)

\subsection*{Linearity And Superposition}
Wave functions add, not probabilities
\\
If $\psi_1$ and $\psi_2$ are two solutions(that is, two wave functions that satisfy the equation), then

\begin{equation}
    \psi=a_1\psi_1+a_2\psi_2
\end{equation}

\begin{equation}
    P_1=|\psi_1|^2=\psi_1*\psi_2
\end{equation}

\begin{equation}
    P_2=|\psi_2|^2=\psi_2*\psi_2
\end{equation}

\begin{equation}
    \psi=\psi_1+\psi_2
\end{equation}

Probability Density at the screen:

\begin{equation}
    P=|\psi|^2=|\psi_1+\psi_2|^2=(\psi_1^*+\psi_2^*)(\psi_1+\psi_2)
\end{equation}

\begin{equation}
    =\psi_1^*\psi_1+\psi_2^*\psi_2+\psi_1^*\psi_2+\psi_2^*\psi_1
\end{equation}

\begin{equation}
    =P_1+P_2+\psi_1^*\psi_2+\psi_2^*\psi_1
\end{equation}

\section{Expectation Value}

Once Schrödinger’s equation has been solved for a particle in a given physical situation, the resulting wave function $\psi$(x, y, z, t) contains all the information about the
particle that is permitted by the uncertainty principle. Expectation values$<x>$ of the position of a
particle confined to the $x$ axis that is described by the wave function $\psi$(x, t).\vspace{6pt}
\\
The probablity is

\begin{equation}
    P_i=|\psi|^2dx
\end{equation}
\\
Expectation Value of the position of the single particle is

\begin{equation}
    <x>=\frac{\int_{-\infty}^{\infty} x|\psi|^2dx }{\int_{-\infty}^{\infty} |\psi|^2dx }
\end{equation}
\\
If $\psi$ is a normalized wave function, the denominator of Eq. (5.18) equals the probability that the particle exists somewhere between x=$-\infty$ and x=$\infty$ and therefore
has the value 1.

\begin{equation}
    <x>=\int_{-\infty}^{\infty} x|\psi|^2dx 
\end{equation}

Expectation value of Potential Energy$(G(x))$

\begin{equation}
    <G(x)>=\int_{-\infty}^{\infty}G(x)|\psi|^2dx 
\end{equation}

The expectation value $<p>$ for momentum cannot be calculated this way.

\section{Operators}
\begin{equation}
    E\psi=i\hbar\frac{\delta\psi}{\delta t}
\end{equation}
\\
An operator tells us what operation to carry out on the quantity that follows it.
\\
Equation (24) was on the postmark
used to cancel the Austrian postage stamp issued to commemorate the 100th
anniversary of Schrödinger’s birth.
It is customary to denote operators by using a caret, so that $\vu*{p}$ is the operator that
corresponds to momentum p and $\vu*{E}$ is the operator that corresponds to total energy E.

\subsection*{Momentum Operator}
\begin{equation}
    \vu*{p}=\frac{\hbar}{i}\frac{\delta}{\delta x}
\end{equation}

\subsection*{Total-Energy Operator}
\begin{equation}
    \vu*{E}=i\hbar\frac{\delta}{\delta t}
\end{equation} \vspace*{20pt}
\\
To support the validity of operator equation. The equation $E=KE+U$ for total energy of a particle can be replaced by with the operator equation

\begin{equation}
    \vu*{E}=\vu*{KE}+\vu*{U}
\end{equation}

\begin{equation}
    \vu*{KE}=\frac{\vu*{p^2}}{2m}=\frac{1}{2m}(\frac{\hbar}{i}\frac{\delta^2\psi}{\delta x^2})^2=-\frac{\hbar^2}{2m}\frac{\delta^2}{\delta x^2}
\end{equation}
\\
Putting value of Equation (28) in Equation (27) and multiplying be $\psi$, we get 

\begin{equation}
    i\hbar \frac{\delta \psi}{\delta t}=-\frac{\hbar^2}{2m}\frac{\delta^2\psi}{\delta x^2}+U\psi
\end{equation}
\\
which is Schrödinger’s equation.


\subsection*{Operators $\&$ Expectation Values}

Expectation Value for p
\begin{equation}
    <p>=\int_{-\infty}^{\infty}\psi^*\vu*{p}\psi dx=\int_{-\infty}^{\infty}\psi^*(\frac{\hbar}{i}\frac{\delta}{\delta x})\psi dx=\frac{\hbar}{i}\int_{-\infty}^{\infty}\psi^*\frac{\delta \psi}{\delta x}dx    
\end{equation}
\\
and the expectation value for E
\begin{equation}
    <p>=\int_{}^{\infty}\psi^*\vu*{E}\psi dx=\int_{}^{\infty}\psi^*(i\hbar\frac{\delta}{\delta t})\psi dx=i\hbar\int_{}^{\infty}\psi^*\frac{\delta \psi}{\delta x}dx 
\end{equation}

\subsection*{Expectation Value of an Operator}

\begin{equation}
    <G(x,p)>=\int_{-\infty}^{\infty}\psi^*\vu*{G}    
\end{equation}

\section*{Schrödinger’s Equation: Steady-State Form}
One-dimensional wave function $\psi$ of an unrestricted particle may be written  
\begin{equation}
    \psi=Ae^{-(i/h)(Et-px)}=Ae^{(iE/h)t}e^{+(ip/\hbar)x}=\psi e^{-(iE/\hbar)t}
\end{equation}
\\
Steady-State form of Schrödinger's equation
\begin{equation}
    \frac{\delta^2 \psi}{\delta x^2}+\frac{2m}{\hbar}(E-U)\psi=0
\end{equation}

\subsection*{Eigenvalues and Eigenvalues}
The values of energy En for which Schrödinger's steady-state equation can be solved are called eigenvalues and the corresponding wave functions $\psi_n$ are called eigenfunctions. \vspace{20pt}
\\
Eigenvalue Equation
\begin{equation}
    \vu*{G}\psi_n=G_n\psi_n
\end{equation}

\subsection*{Hamilton Operator}
\begin{equation}
    \vu*{H}=-\frac{\hbar^2}{2m}\frac{\delta^2}{\delta x^2}+U
\end{equation}
and is called the Hamiltonian operator because it is reminiscent of the Hamiltonian
function in advanced classical mechanics, which is an expression for the total energy
of a system in terms of coordinates and momenta only. Evidently the steady-state
Schrödinger equation can be written simply as
\subsection*{Schrödinger's equation}
\begin{equation}
    \vu*{H}\psi_n=E_n\psi_n
\end{equation}

\break

\section*{Particle In A Box}
To solve Schrödinger's equation, even in its simpler steady-state form, usually requires elaborate mathematical techniques.Quantum mechanics is the theoretical structure
whose results are closest to experimental reality, we must explore its methods and applications to understand modern physics.\vspace{4pt}
\\
The simplest quantum-mechanical problem is that of a particle trapped in a box
with infinitely hard walls.
\\
From a formal point of view the potential energy U of the particle is infinite on both sides of
the box, while U is a constant—say 0 for convenience—on the inside (Fig. 5.4). Because
the particle cannot have an infinite amount of energy, it cannot exist outside the box,
and so its wave function $\psi$ is 0 for $x\le0$ and $x\ge L$. Our task is to find what $\psi$ is
within the box, namely, between x = 0 and x = L.
\end{document}